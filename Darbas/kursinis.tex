\documentclass{VUMIFInfKursinis}
\usepackage{algorithmicx}
\usepackage{algorithm}
\usepackage{algpseudocode}
\usepackage{amsfonts}
\usepackage{amsmath}
\usepackage{bm}
\usepackage{color}
% \usepackage{hyperref}  % Nuorodų aktyvavimas
\usepackage{url}


% Titulinio aprašas
\university{Vilniaus universitetas}
\faculty{Matematikos ir informatikos fakultetas}
\department{Programų sistemų katedra}
\papertype{Kursinis darbas}
\title{Objektų atpažinimas ir sekimas kompiuterinės tomografijos vaizduose}
\titleineng{Object detection and tracking in computed tomography}
\status{3 kurso Programų sistemų studentas}
\author{Paulius Milmantas}
\supervisor{Linas Petkevičius}
\date{Vilnius \\ \the\year}

% Nustatymai
% \setmainfont{Palemonas}   % Pakeisti teksto šriftą į Palemonas (turi būti įdiegtas sistemoje)
\bibliography{bibliografija}

\begin{document}
\maketitle

\tableofcontents

\section{Dirbtinio neuroninio tinklo sudėtis}
\subsection{Bazinė struktūra}
Daugelį dirbtinių neuronų tinklų sudaro panašios struktūrinės dalys:
\begin{enumerate}
  \item Įvesties sluoksnis: tai dalis kuri priima įvestį ir perduoda kitiems sluoksniams.
  \item Išvesties sluoksnis: tai dalis, kuri naudoja aktyvacijos funkciją kuri grąžina galutinį tinklo rezultatą: tikimybių rinkinį, kuris parodo kokia tikimybė, kad objektas atitinka tam tikrą klasę.
  \item Paslėptas sluoksnis: perduoda svorius iš praeito sluoksnio į sekantį.
  \item Susijungimai ir svoriai: tarp kiekvieno neurono, kuris yra susijungęs, turi savo svorį, pagal kurį yra pakeičiama perduodama reikšmė.
  \item Aktyvacijos funkcija: tai funkcija, kokia turi būti neurono išvestis.
  \item Mokymosi taisyklė: apibrėžia, kaip tinkle keičiasi svoriai, kad tinklas išvestų norimus rezultatus.
\end{enumerate}

\subsection{Poskyris}
Citavimo pavyzdžiai: cituojamas vienas šaltinis \cite{PvzStraipsnLt}; cituojami
keli šaltiniai \cite{PvzStraipsnEn, PvzKonfLt, PvzKonfEn, PvzKnygLt, PvzKnygEn,
PvzElPubLt, PvzElPubEn, PvzMagistrLt, PvzPhdEn}.

\subsubsection{Skirsnis}
\subsubsubsection{Straipsnis}
\subsubsection{Skirsnis}
\section{Skyrius}
\subsection{Poskyris}
\subsection{Poskyris}

\sectionnonum{Išvados}
Išvadose ir pasiūlymuose, nekartojant atskirų dalių apibendrinimų,
suformuluojamos svarbiausios darbo išvados, rekomendacijos bei pasiūlymai.

\printbibliography[heading=bibintoc] % Literatūros šaltiniai aprašomi
% bibliografija.bib faile. Šaltinių sąraše nurodoma panaudota literatūra,
% kitokie šaltiniai. Abėcėlės tvarka išdėstoma tik darbe panaudotų (cituotų,
% perfrazuotų ar bent paminėtų) mokslo leidinių, kitokių publikacijų
% bibliografiniai aprašai (šiuo punktu pasirūpina LaTeX). Aprašai pateikiami
% netransliteruoti.

\appendix  % Priedai
% Prieduose gali būti pateikiama pagalbinė, ypač darbo autoriaus savarankiškai
% parengta, medžiaga. Savarankiški priedai gali būti pateikiami kompiuterio
% diskelyje ar kompaktiniame diske. Priedai taip pat vadinami ir numeruojami.
% Tekstas su priedais siejamas nuorodomis (pvz.: \ref{img:mlp}).

\section{Niauroninio tinklo struktūra}
\begin{figure}[H]
    \centering
    \caption{Paveikslėlio pavyzdys}   % Antraštė įterpiama po paveikslėlio
    \label{img:mlp}
\end{figure}


\section{Eksperimentinio palyginimo rezultatai}
% tablesgenerator.com - converts calculators (e.g. excel) tables to LaTeX
\begin{table}[H]\footnotesize
  \centering
  \caption{Lentelės pavyzdys}    % Antraštė įterpiama prieš lentelę
  {\begin{tabular}{|l|c|c|} \hline
    Algoritmas & $\bar{x}$ & $\sigma^{2}$ \\
    \hline
    Algoritmas A  & 1.6335    & 0.5584       \\
    Algoritmas B  & 1.7395    & 0.5647       \\
    \hline
  \end{tabular}}
  \label{tab:table example}
\end{table}

\end{document}
